\documentclass[11pt]{article}

\usepackage[letterpaper,top=2cm,bottom=2cm,left=2.5cm,right=2.5cm,marginparwidth=1.75cm]{geometry}

%packages
\usepackage{amsmath}
\usepackage{graphicx}
\usepackage{amssymb}
\usepackage{algorithm}
\usepackage{algpseudocode}
\usepackage{color,soul}
\usepackage{mathtools}
% \usepackage{subfig}
\usepackage{subcaption}
\usepackage{caption}

% Edits by John:
% Use hyperref when you're referencing anything - in particular, use the \autoref{} command - it's great. One exception: anything in mathmode should be referenced using \eqref{} instead; \autoref{} calls all mathmode objects "equations", even when they're not equations (definitions, inequalities, propositions, statements, etc.), so it's better to use the \eqref{} function.
\usepackage[colorlinks, linkcolor=blue, citecolor=blue, urlcolor=blue]{hyperref}
% Use the align (or similar) environment, rather than built-in latex commands for math statements. That way, things can be aligned properly and equations will be numbered and referencable. The equation below assigns equation numbers based upon the current section.

\numberwithin{equation}{section}
% Use the amsthm environents to define theorems, remarks, definitions, etc., with commands of the form \begin{definition}[DEFINITION ITTLE]
\usepackage{amsthm}% provides the environments
\theoremstyle{definition}% provides a style for definitions - this affects all downstream \newtheorem statements until \theoremstyle is used again.
\newtheorem{theorem}{Theorem}
\newtheorem{definition}{Definition}[section]% numbers definitions within sections
% The singular of "matrices" is "matrix", not "matrice" - the abnormal singular-plural pair is an importation from Latin.
% Use \url{} for hyperlinks (I changed the pytorch link). I think this is included in hyperref, but it may be in base LaTeX.

\DeclarePairedDelimiter\ceil{\lceil}{\rceil}
\DeclarePairedDelimiter\floor{\lfloor}{\rfloor}
\newcommand{\pluseq}{\mathrel{+}=}
\newcommand{\asteq}{\mathrel{*}=}

\usepackage[dvipsnames]{xcolor}
\usepackage[many]{tcolorbox}

\definecolor{lavender}{RGB}{214, 111, 208}
\colorlet{lavender}{lavender!50}

\newcommand{\hlinfo}[1]{{\sethlcolor{lavender}\hl{#1}}}
\newcommand{\note}[1]{\textcolor{red}{[#1]}}

\usepackage{tikz}

\setlength\parindent{0pt}

\begin{document}

\noindent
\begin{center}
    \section*{\centering{Week 3 Notes}}
    \subsection*{\centering{\emph{Lectures 5 \& 6}}}
    \emph{University of Massachusetts Amherst}, CS389
\end{center}

\section{Multilayer Perceptrons (lecture 5)}

\subsection{Non-linearities}

Examples of non-linearity functions can be refered below

\subsubsection{Sigmoid}

\begin{align}
    \text{sig}(x) = \frac{1}{1+e^{-x}}
\end{align}

The derivative is:
\begin{align}
    \frac{d}{dx}\text{sig}(x) = \text{sig}(x) \cdot (1-\text{sig}(x))
\end{align}

\subsubsection{Tanh}

\begin{align}
    \text{tanh}(x) = \frac{\text{sinh}(x)}{\text{cosh}(x)} = \frac{e^{x}-e^{-x}}{e^{x}+e^{-x}}
\end{align}

The derivative is:
\begin{align}
    \frac{d}{dx} \text{tanh}(x) = 1 - \text{tanh}^{2}(x)
\end{align}

\subsubsection{ReLU}

\begin{align}
    \text{ReLU}(x) = \text{max}(0, x)
\end{align}

Derivative is:

\begin{align}
    \frac{d}{dx}\text{ReLU}(x) = \begin{cases}
        0 & x < 0 \\
        1 & \text{otherwise}
    \end{cases}
\end{align}

\section{Backpropagation (lecture 5)}

\begin{tcolorbox}[title=Derivation:,colframe=purple,colback=blue!5!white,arc=0pt,fonttitle=\bfseries]
    The algebraic system define over operator $\star$ , which is \emph{closed} and \emph{associative} is called SEMIGROUP.
\end{tcolorbox}


\subsection{Code implementation (lecture 6)}



\begin{thebibliography}{2}
    \bibitem{DrCoop} Cooper.

\end{thebibliography}

\end{document}
